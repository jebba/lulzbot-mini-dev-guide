%
% Software.tex
% Free Software
%
% LulzBot Easy TAZ Mini Developer's Guide
%
% Copyright (C) 2014 Aleph Objects, Inc.
%
% This document is licensed under the Creative Commons Attribution 4.0
% International Public License (CC BY-SA 4.0) by Aleph Objects, Inc.
%
\section{Intro}
This chapter covers the software that runs on the EZGNU embedded hardware board.


\section{Bootloader}
U-boot is the bootloader. The linux-sunxi fork is used as a base.


\section{Linux Kernel}

\subsection{Intro}

This is the development archive of software used in version Camillia of the
LulzBot Easy TAZ Mini 3D Printer by Aleph Objects, Inc.

All software is copyright by the respective upstream authors, except new
additions, which are copyright by Aleph Objects, Inc. under the GPLv3.



\subsection{Kernel Branch}

We will be using Linux Sunxi as the base source code for the kernel.


There are various kernel branches that could be used as a base:

\begin{itemize}
  \item{Mainline Linus -- This doesn't have much of the latest/greatest for A20.

  \url{git://git.kernel.org/pub/scm/linux/kernel/git/torvalds/linux.git}}

  \item{Linksprite -- This is optimized for pcduino3, which is very similar to the
  board we'll be using. It uses non-free software, such as the Mali driver.
  Using it as a base will be a bit messy. It also does non-standard patching
  in the build process, instead of just committing everything to git. The
  build system overall is nice and does more than just the kernel. It depends
  upon the non-free Allwinner livesuit image tools. We will use another tree
  as a base, but will pick select drivers from this one.

  \url{git://github.com/pcduino/a20-kernel.git}}

  \item{Linux Sunxi -- This is the main kernel branch for the Sunxi platform built
  upon the Allwinner ARM chips. It is actively maintained. The Easy TAZ Mini
  core is built upon their various archives. Main website:

  \url{http://linux-sunxi.org/}

  \url{git://github.com/linux-sunxi/linux-sunxi.git}}

\end{itemize}



\subsection{Kernel Version}

We will be using the main linux-sunxi git repo, using the sunxi-3.4 branch as
the main base for the Linux kernel. The latest version is 3.4.90.

There are also various kernel versions we could chose from. Some options:

\begin{itemize}
  \item sunxi-next -- This looks pretty good. Worth exploring more.

  \item sunxi-devel -- Probably not.

  \item 3.4.79+ -- This is the kernel that gets built by the default a20-kernel
  archive from Linksprite. Known to work. Has non-free software.

  \item 3.14 -- This is the latest version from the sunxi-3.14 branch of the main
  linux-sunxi kernel. It has not seen as much development or testing as
  sunxi-3.4. It does have -sunxi patches and is based on a much more recent
  upstream kernel. The one test kernel I built didn't fully boot, but it
  likely can be made to work without too much pain. As it hasn't seen as much
  real-world usage, sunxi-3.4 is preferred.

  \item 3.4.90 -- This is the latest version from the sunxi-3.4 branch of the main
  linux-sunxi kernel. This is known to work. We will likely use a version of
  this kernel.
\end{itemize}

\subsection{Building the Kernel}


Quickie overview:

\begin{enumerate}
  \item {Clone the kernel archive we want to use:
  
  \verb|git clone| \url{git://github.com/linux-sunxi/linux-sunxi.git}
  
  \verb|cd linux-sunxi|
  }

  \item{Checkout the branch we want:
  
  \verb|git checkout sunxi-3.4|
  }

  \item{Copy over camillia kernel config (check for newer version):
  
  \verb|wget| \url{http://devel.lulzbot.com/Easy_TAZ_Mini/camellia/software/current/OUTPUT/eztaz_defconfig/eztaz_defconfig-ao11}
  
  \verb|cp eztaz_defconfig-ao11 linux-sunxi/.config|
  }

  \item{Go into kernel config and make whatever changes:

  \begin{verbatim}
  LOADADDR=0x40008000 make -j1 ARCH=arm         \
  CROSS_COMPILE=arm-linux-gnueabihf- menuconfig
  \end{verbatim}
  }

  \item{Copy that \verb|.config| somewhere as backup
  
  \verb|cp .config ../OUTPUT/eztaz_defconfig/eztaz_defconfig-ao999|
  }

  \item{Build the kernel uImage and modules:
  
  \begin{verbatim}
  LOADADDR=0x40008000 make -j4 ARCH=arm         \
  CROSS_COMPILE=arm-linux-gnueabihf- uImage modules dtbs
  \end{verbatim}
  }

  \item{Install kernel modules to the \texttt{OUTPUT} dir:

  \begin{verbatim}
  LOADADDR=0x40008000 make -j4 ARCH=arm         \
  CROSS_COMPILE=arm-linux-gnueabihf-            \
  INSTALL_MOD_PATH=../OUTPUT modules_install
  \end{verbatim}
  }

\end{enumerate}


Misc build commands

\begin{itemize}
  \item{This will build the default sun7i (pcduino3) kernel (example):
  
  \begin{verbatim}
  LOADADDR=0x40008000 make -j4 ARCH=arm		\
  CROSS_COMPILE=arm-linux-gnueabihf-		\
  sun7i_defconfig
  \end{verbatim}
  }

  \item{If you need to clean up:
  
  \begin{verbatim}
  make clean ARCH=arm CROSS_COMPILE=arm-linux-gnueabihf-
  \end{verbatim}
  }

  \item{You probably don't need to:
  
  \begin{verbatim}
  make mrproper ARCH=arm CROSS_COMPILE=arm-linux-gnueabihf-
  \end{verbatim}
  }
\end{itemize}


\section{Core OS}
\begin{itemize}
  \item{Debian Wheezy (stable) release}
  \item{armhf architecture}
\end{itemize}


\section{Graphical User Interface}

\begin{itemize}
  \item{fb}
  \item{X Windows}
  \item{EZTAZ}
  \item{Octoprint Web Interface}
\end{itemize}

\section{3D Object Processing}

\begin{itemize}
  \item{Slic3r}
  \item{Meshlab}
\end{itemize}

